\documentclass{beamer}
\usetheme{Hannover}
% \setbeamertemplate{footline}[frame number]
\usepackage[]{graphicx}
\usepackage[]{color}
\usepackage{alltt}
\usepackage{amsmath}
\usepackage{amssymb}
\usepackage{breqn}
\usepackage{amsthm}
\usepackage{subcaption}
\usepackage{tikz}
\usepackage{setspace}

\DeclareMathOperator{\diag}{diag}
\DeclareMathOperator{\pen}{pen}
\DeclareMathOperator{\relu}{ReLU}
\DeclareMathOperator{\id}{id}
\DeclareMathOperator{\wdr}{WDR}
\DeclareMathOperator{\kl}{KL}

\newcommand{\mytitle}{Survey on Regularization Methods in Continual Learning}
\newcommand{\myname}{Jörg Schantz}
\newcommand{\mysupervisor}{Dr. Julian Rodemann}


\title{\mytitle}
\author{Jörg Schantz}
\institute{Ludwig-Maximilian Universität München}
\date{April 4, 2025}



\begin{document}
\begin{frame}
    \maketitle
    \centering
    Supervised by \mysupervisor
\end{frame}
\begin{frame}{Outline}
    \tableofcontents
\end{frame}

\section{Introduction}

\begin{frame}{Intro}

\end{frame}

\section{Framework}

\begin{frame}{Framework}
	
\end{frame}

\section{Stability-Plasticity-Dilemma}

\begin{frame}{Stability-Plasticity Dilemma}
	
\end{frame}

\section{Measuring Feature Importance}

\begin{frame}{Measuring Feature Importance}
	
\end{frame}

\section{Linear Continual Learning}

\begin{frame}{Linear Continual Learning}
	
\end{frame}

\section{Other Squared Penalties}

\begin{frame}{Other Squared Penalties}
	
\end{frame}

\section{LASSO Inspired Penalties}

\begin{frame}{LASSO Inspired Penalties}
	
\end{frame}

\section{Output Penalties}

\begin{frame}{Output Penalties}
	
\end{frame}

\section{Conclusion}

\begin{frame}{Conclusion}
	
\end{frame}

\section*{Sources}

\begin{frame}{Sources}
	
\end{frame}
\end{document}
