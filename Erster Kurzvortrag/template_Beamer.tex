\documentclass{beamer}
\usetheme{default}
\setbeamertemplate{footline}[frame number]
\title{Kurzvortrag: Survey on Continual Learning}
\author{Jörg Schantz}
\date{26. November 2024}
\begin{document}
\begin{frame}[plain]
    \maketitle
\end{frame}
\begin{frame}
    \frametitle{Aktueller Gliederungsentwurf}
    \begin{enumerate}
    	\item Introduction
    	\item Framework for CL
    	\begin{enumerate}
    		\item Evaluation Metrics
    		\item Distribution Drift and Bayes
    	\end{enumerate}
    	\item General Approaches
    	\begin{enumerate}
    		\item Replay
    		\item Optimization
    		\item Representation
    		\item Architecture
    		\item Regularization
    	\end{enumerate}  	
    	\item Regularization Approaches
    	\begin{enumerate}
    		\item Parameter Space
    		\item Function Space
    	\end{enumerate}
    	\item Conclusion
 	\end{enumerate}
\end{frame}
\begin{frame}
	\frametitle{2. Framework}
	Tasks $t = 1, ..., T$ und Sampleset $\mathcal{D} = \left\lbrace \mathcal{D}_1, ..., \mathcal{D}_T\right\rbrace$
	\\
	\vspace{0.5cm}
	$\mathcal{D}_t = \lbrace (x_i^{(t)},y_i^{(t)}) \rbrace_{i=1}^{n_t}$ mit $n_t$ als die t-te Stichprobengröße
	\\
	\vspace{0.5cm}
	$\mathcal{D}_t$ folgen den Verteilungen $\mathbb{D}_t = p(X_t, Y_t)$ 
	\\
	\vspace{0.5cm}
	Model 
	$p(\mathcal{D}_{1:k}|\theta) = \prod_{t=1}^{k}p(\mathcal{D}_t|\theta)$
\end{frame}
\begin{frame}
	\frametitle{2.1 Evaluation Metrics}
	Average Accuracy $AA_k=\frac{1}{k}\sum_{i=1}^{k}a_{k,i}$
	\\
	\vspace{0.5cm}
	Backward Transfer $BWT_k = \frac{1}{k-1}\sum_{i=1}^{k-1}(a_{k,i}-a_{i,i})$
	\\
	\vspace{0.5cm}
	Forward Transfer $FWT_k = \frac{1}{k-1}\sum_{i=2}^{k}(a_{i,i}-\tilde{a}_i)$
	\\
	\vspace{1cm}
	$\tilde{a}_{k,i}$ Accuracy des Models p($\mathcal{D}_{1:k}|\theta)$ auf Testset $i$
	\\
	$\tilde{a}_i$ Accuracy des Models $p(\mathcal{D}_i|\theta)$
\end{frame}
\begin{frame}
	\frametitle{2.2 Distribution Drift and Bayes}
	Verteilungen alter und neuer Tasks müssen erhalten bleiben
	\\
	\vspace{0.5cm}
	Speichern oder generieren alter Daten
	\\
	\vspace{0.5cm}
	Formulierung eines Prior basierend auf dem bisher Gelernten (alter Posterior)
	als Laplace Approximation oder Variational Inference
\end{frame}
\begin{frame}
	\frametitle{3.1 Replay}
	\textbf{Experience Replay} speichert kleine Menge alter Samples
	\\
	Auswahl: feste Menge, nah am Mean, Erhaltung des Gradient
	\\
	\vspace{0.5cm}
	\textbf{Generative Replay}: zusätzliches Generativ Model; $X_k$ und $X'_{1:k-1}$ ja nach Gewichtung gemischt
\end{frame}
\begin{frame}
	\frametitle{3.2 Optimization}
	Parameter-Update durch alte Gradient Descent Richtung kontrolliert.
	\\
	Orthogonal; Dynamische Priorisierung
	\\
	\vspace{0.5cm}
	Anpassung des "loss-landscape"
	\\
	\vspace{0.5cm}
	Meta-learning als Versuch Task-spezifische Bias zu lernen
\end{frame}
\begin{frame}
	\frametitle{3.3 Representation}
	Pre-Training methoden und Self-Supervised Learning
	\\
	Herausforderung ist das Pre-Training flexibel einzusetzen 
\end{frame}
\begin{frame}
	\frametitle{3.4 Architecture}
	Aufteilung der Parameter in "Task-spezifisch" und "geteilt"
	
\end{frame}
\begin{frame}
	\frametitle{4.1 Parameter Space}
	Gewichtung der Parameter basierend auf Fisher Information mit einer idR quadratischen Loss-Funktion
	\\
	Über FI direkt, Total-Loss, Änderung der Prediction
	\\
	\vspace{0.5cm}
	Andere sind: Variational Inference, Forgetting Rate um den Prior zu regulieren
\end{frame}
\begin{frame}
	\frametitle{4.2 Function Space}
	Reguliert den Prediction Output
	\\
	Teacher / Student Beziehung  in Kombination mit Knowledge Distillation 
	\vspace{0.5cm}
	Unterschiede in Vergleichsgröße der KD (alte Samples, generiert, neues Sample, unlabeled Data), deswegen oft in Kombination mit Replay
\end{frame}
\begin{frame}
	\frametitle{Conclusion}
	Großes Fragezeichen... 
\end{frame}
\end{document}
