\documentclass[12pt, a4paper]{report}
\usepackage[a4paper, left=3.5cm, right=3cm, top=3cm,bottom=3cm]{geometry}
\title{Exposé\\
	\Large{Survey on Continual Learning}}
\author{Jörg Schantz (ru23cor)}
\date{November 2024}
\begin{document}
	\maketitle
	\renewcommand{\labelenumii}{\arabic{enumi}.\arabic{enumii}}
	\renewcommand{\labelenumiii}{\arabic{enumi}.\arabic{enumii}.\arabic{enumiii}}
	\renewcommand{\labelenumiv}{\arabic{enumi}.\arabic{enumii}.arabic{enumiii}.\arabic{enumiv}}
\section*{Idea}
	The idea of this thesis is to give a good overview of the machine learning (ML) field continual learning (CL) but through the lens of a statistician. This is to be achieved by gathering various papers on the topic and breaking down their statistical foundations. These should cover a broad spectrum of the possibilities in CL e.g. the different methodologies or criteria.\\
	Since I enjoy coding and fiddling around with practical models more than their theory I could imagine training a linear continual learner with different regularization methods or something in the likes of for the last part of the thesis. But I am a little skeptical whether this is feasible or not. Over the last year I have also developed an interest in fraud analysis, but haven't had the time to really dive into this field. I am also thinking about taking this as an opportunity to educate myself and connect the two topics. But have no  Alternatively, Dr. Rodemann suggested to deeper analyse a singular paper.\\
\section*{Potential structure}
	\begin{enumerate}
		\item Differences between ML and Statistical Modelling
		\item The challenges of continual learning 
		\begin{enumerate}
			\item Distribution Shift
			\item Catastrophic Forgetting
		\end{enumerate}
		\item Defining a continual learner
		\item 
	\end{enumerate}
	gewünschter Abgabetermin:\\
\section*{Progress of my Studies}
	Now about the current state of my studies. I started this degree in	2017 with a minor in Insurance and Risk Management. After the first year I changed my minor to computer science. I am still dealing with a couple of mental issues, hence the long duration, but I am determined to finish my degree. This semester I intend to take my last two exams in GRM and I2ML. Both courses I have previously heard and solved most exercise sheets but never taken the final exam.\\
	In my third semester I have taken the WP course "Stichprobentheorie" and in my eighth "Einführung in die Biometrie". I2ML is my last WP course that I have to pass.\\
	The BA Seminar was together with Dr. André Klima and Dr. Georg Schollmeyer in 2020. Latter supervised my project, Visualisierung von Verteilungsunsicherheiten.
\section*{Familiarization}
	
\end{document}