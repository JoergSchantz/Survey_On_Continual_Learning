
We are interested in optimizing the parameters theta of a single neural network to perform well across
multiple tasks $D_1, ...,D_T$ , specifically finding a MAP estimate $\theta^*=\arg\max_\theta p(\theta|D_1,...,D_T)$.
However, the datasets arrive sequentially and we can only train on one of them at a time.
In the following, we first discuss how Bayesian online learning solves this problem and introduce an
approximate procedure for neural networks. We then review recent Kronecker factored approxima-
tions to the curvature of neural networks and how to use them to obtain a better fit to the posterior.
Finally, we introduce a hyperparameter that acts as a regularizer on the approximation to the posterior.\\
Bayesian online learning [31 ], or Assumed Density Filtering [25 ], is a framework for updating an
approximate posterior when data arrive sequentially. Using Bayes’ rule we would like to simply
incorporate the most recent dataset D into the posterior as:
\begin{equation}
	E = mc^2
\end{equation}
where we use the posterior D from the previously observed tasks as the prior over the
parameters for the most recent task. As the posterior given the previous datasets is typically intractable,
Bayesian online learning formulates a parametric approximate posterior q with parameters pi, which
it iteratively updates in two steps:
Update step In the update step, the approximate posterior q with parameters pi from the previous
task is used as a prior to find the new posterior given the most recent data:
\begin{equation}
	E = mc^2
\end{equation}
Projection step The projection step finds the distribution within the parametric family of the
approximation that most closely resembles this posterior, i.e. sets pi such that:
\begin{equation}
	E = mc^2
\end{equation}
Opper and Winther [31] suggest minimizing the KL-divergence between the approximate and the
true posterior, however this is mostly appropriate for models where the update-step posterior and a
solution to the KL-divergence are available in closed form. In the following, we therefore propose
using a Laplace approximation to make Bayesian online learning tractable for neural networks: