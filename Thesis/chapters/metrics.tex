%%%%%%%%%%%%%%%%%%% Metrics %%%%%%%%%%%%%%%%%%%
Intro.\\
In the following each sample $D_t = (X_t, Y_t)$ is divided into a training split $D_t^{(train)} = (X_t^{(train)}, Y_t^{(train)})$ and a testing split $D_t^{(test)} = (X_t^{(test)}, Y_t^{(test)})$. The chosen splitting method is arbitrary. The training process for each sample will be conducted with $D_t^{(train)}$ and evaluation with $D_t^{(test)}$.\\
\cite{LW} mention different measures for model performance, stability and plasticity. I will focus on the dynamic forms given by \cite{díazrodríguez2018dontforgetforgettingnew}, because they are adapted for in training use i.e. they represent a model's current state after the $t$-th training step.\\
\textit{Accuracy} $\mathbf{A}$ represents a models performance i.e. how well the predictions $\hat{Y}_t^{(test)}$ align with the true values of $Y_t^{(test)}$ for a metric $\mu$ . When $A_{i,k}$ is the accuracy measured on the $k$-th test split after the $i$-th training step, then
\begin{equation}
	\mathbf{A} = \frac{2}{t(t-1)}\sum_{i \geq k }^{t} A_{i,k}
\end{equation}
is the average accuracy after the $t$-th training step over all test splits $D_k^{(test)}, k <= t$ \cite{díazrodríguez2018dontforgetforgettingnew}.\\
\textit{Backward Transfer} $\mathbf{BWT}$ evaluates a models stability \cite{LW}. The metric quantifies the influence of learning sample $D_{t+1}^{(train)}$ has on the performance over test sample $D_t^{(test)}$ \cite{lopezpaz2022gradientepisodicmemorycontinual}. Given, the above mentioned, individual \textit{Accuracy} scores $A_{i,k}$
\begin{equation}
	\mathbf{BWT} = \frac{2}{t(t-1)} \sum_{i=2}^{t}\sum_{k=1}^{i-1}(A_{i,k}-A_{k,k})
\end{equation}
is the average backward transfer after the $t$-th training step \cite{díazrodríguez2018dontforgetforgettingnew}. Note that $\mathbf{BWT}$ can be negative. This property captures (catastrophic) forgetting \cite{LW}.\\
\textit{Forward Transfer} $\mathbf{FWT}$ is a metric for model plasticity \cite{LW}. Complementary to BWT, \textit{Forward Transfer} measures how previous training steps influence the current one. Again the individual \textit{Accuracy} scores are the basis for this evaluation metric. The average influence of old training steps on the model performance after the $t$-th step:
\begin{equation}
	\mathbf{FWT} = \frac{2}{t(t-1)}\sum_{i < k }^{t} A_{i,k}
\end{equation}
\cite{díazrodríguez2018dontforgetforgettingnew}.\\
Another metric that directly measures the relationship between stability and plasticity is presented in \cite{mirzadeh2020understandingroletrainingregimes}. The authors use the maximum eigenvalue of the loss' Hessian $\lambda^{max}$ to describe the width of their approximation of the loss' minimum. They hypothesize that the \textit{wideness} of this minimum correlates with the forgetting rate of the respective model.\\
Given $W_t^*$ and $W_{t+1}^*$ the optimal parameters after learning the $t$-th and $t+1$-th task and $L_t(\cdot)$ and $L_{t+1}(\cdot)$ the corresponding loss functions. \citeauthor{mirzadeh2020understandingroletrainingregimes} formulate the upper bound
\begin{equation}
	F_t = L_t(W_{t+1}^*) - L_t(W_t^*) \approx \frac{1}{2}{\Delta W}^\top \nabla^2 L_t(W_t^*) \Delta W \leq \frac{1}{2}\lambda_t^{max}\lVert \Delta W \rVert^2
\end{equation}
for the forgetting $F_t$ of the $t$-th task. They approximate $L_t(W_{t+1}^*)$ around $W_t^*$ with a second order Taylor approximation, where $\nabla^2$ is the Hessian for $L_t$ and $\Delta W$ the difference between $W_{t+1}^*$ and $W_t^*$. They argue that the loss can be approximated this way, because of its almost convex path around the minimum, for models that have more observations per sample than parameters.\\
Further, ${\Delta W}$ is dependent on the training process of the $t+1$-th task, which depends on the random sample it is trained on, so one can view the differences in parameters as a random vector, that follows some distribution parameterized by the eigenvalues of $\nabla^2 L_t(W_t^*)$ \cite{mirzadeh2020understandingroletrainingregimes}.\\

Controlling the distance of the weights seems to be the key to mitigating forgetting...